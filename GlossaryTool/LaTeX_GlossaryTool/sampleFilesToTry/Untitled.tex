\documentclass{article}
\documentclass{article}
\usepackage[colorlinks]{hyperref}
\usepackage{glossaries}
 
\makeglossaries
  
\title{How to create a glossary}
\author{ }
\date{ }
 
\begin{document}
\maketitle
 
The \Gls{latex} typesetting markup is specially suitable for documents that include \gls{maths}. 

Java was created at Sun Microsystems, Inc., where cofounder William (Bill) Joy led a team of researchers in an effort to create a 
new language that would allow consumer electronic devices to communicate with each other. Work on the language began in 1991, and 
before long the team’s focus changed to a new niche, the World Wide Web. Java was first used on the Web in 1994, and Java’s 
ability to provide interactivity and multimedia showed that it was particularly well suited for the Web.

object
Object
objects
Objects
obj

code
Code
codes
Codes
cd

The difference between the way Java and other programming languages worked was revolutionary. Code in other languages is first 
translated by a compiler into instructions for a specific type of computer. The Java compiler instead turns code into something 
called Bytecode, which is then interpreted by software called the Java Runtime Environment (JRE), or the Java virtual machine. 
The JRE acts as a virtual computer that interprets Bytecode and translates it for the host computer. Because of this, Java code \
can be written the same way for many platforms (“write once, run anywhere”), which helped lead to its popularity for use on the 
Internet, where many different types of computers may retrieve the same Web page. By the late 1990s, Java had brought multimedia 
to the Internet and started to grow beyond the Web, powering consumer devices (such as cellular telephones), retail and financial 
computers, and even the onboard computer of NASA’s Mars exploration rovers. Because of this popularity, Sun created different 
varieties of Java for different purposes, including Java SE for home computers, Java ME for embedded devices, and Java EE for 
Internet servers and supercomputers.
 
\clearpage
 
\printglossaries
 
\end{document}
